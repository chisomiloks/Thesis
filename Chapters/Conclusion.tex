\chapter{Conclusion and Future Work}
\label{chap:conclusions}

\section{Overview}

This chapter provides a description of the contribution of this thesis and also provides recommendation fo future developments that could be implemented in extending the functionalities of the framework.

\section{Conclusion}

The challenge of this thesis was to develop a system that would allow for the secure exchange of EHRs among  multiple parties through the use of potentially untrusted third party services, such as cloud computing. The solution to this challenge would need to be able to provide adequate security and privacy as a result of the sensitive nature health data. It would also be important for the solution to be able to provide effective access control at a granular level. 

To create this solution, we have developed a framework for the sharing of electronic health data. We have placed the core of this framework an effective ABE schemes, a Multi-Authority Ciphertext Policy Attribute Based Encryption with Outsourcing (MACP-ABEwO) scheme. We have developed our scheme by using existing techniques to modify an effective multi-authority scheme \cite{Yang2014}. Our scheme allows for the secure outsourcing of decryption to the Cloud Service Provider (CSP), significantly reducing the computational resources of the end user devices required for accessing data. This is achieved by offloading the more computationally intensive operations of the decryption process to the CSP. This also allows for a significant reduction in the amount of time for decryption as the framework takes advantage of the resources provided by the CSP in terms of computational power.

Our framework allows for the secure exchange of health data not simply among medical institutions, but also with third party partners, such as researchers, who contribute to developments in the area while preserving privacy. Our framework uses symmetric encryption for the security of data and uses our proposed MACP-ABEwO scheme for the encryption of secret keys users need to access data in its original form.

We have developed our framework using existing tools in order to evaluate the performance of our underlying scheme. We have built the different modules and run them locally on a Linux server in order to save costs and evaluated the performance of the algorithms that would typically run on the end user device. We have evaluated the performance for encryption, decryption and revocation and shown how they scale with an increase in the number of Attribute Authorities and attributes. Our results show no negative impact on the encryption and revocation time of the original scheme in comparison with out but shows a significant decrease in the decryption time. This is because, as a result of outsourcing, the decryption time remains constant irrespective of the number of attribute authorities or attributes involved. We also evaluate the security of our scheme and show how it is able to ensure confidentiality, integrity, and authentication, while also preserving privacy.

\section{Future Work}

There are different areas of improvement in relation to our existing framework as described below:

\begin{enumerate}[label=(\arabic*)]
	\item Accountability - Our current framework is limited in its ability to ensure strong levels of accountability as it can only be limited to the group of users who have been granted secret keys for the corresponding attributes which they share with others with the same level of access. A possible future development would be to extend our underlying ABE scheme potentially using similar pairing based short signatures that are being developed with minimal overhead on our framework.
	
	\item Searchable Encryption - Our current setup requires some sort of record in the database to keep track of what type of data are being stored in their encrypted form. Potential future developments would include the extension of our underlying scheme to allow for searching at a more granular level over the encrypted data.
	
\end{enumerate}

In conclusion, our secure privacy preserving cloud based framework for sharing electronic health data is a platform that allows for the sharing of electronic health records. Taking into accounts the challenges that exist with the use of third party storage services and the sensitive nature of the data being shared, our framework provides potential solutions to the existing limits in the functionality of current systems. Our framework is able to do this through the use of an effective Multi-Authority Ciphertext Policy Attribute Based Encryption with Outsourcing (MACP-ABEwO) scheme to provide adequate levels of security, preserve privacy and allows for effective access control at a granular level. 
