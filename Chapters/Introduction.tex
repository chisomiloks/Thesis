\chapter{Introduction}
\label{chap:introduction}

%\chapter{Chapter Title Here} % Main chapter title
%\label{Chapter1} % For referencing the chapter elsewhere, use \ref{Chapter1} 

\section{Motivation}

Secure storage of electronic health data by health institutions such as hospitals is important in order to make this data available to other parties while ensuring that they prevent unauthorized access to user data and protect the privacy of their patients. These institutions typically store this data on physical hardware in a secure location on their premises and so are able to prevent unauthorized access through the use of adequate network security infrastructure. They are also able to prevent authorized users from gaining access beyond their levels of control through the use of proper access control mechanisms which ensures accountability. The deployment of more efficient and modern information technology such as cloud computing extends the trust boundary of the Information Technology (IT) infrastructure of institutions beyond their facilities while giving them multiple advantages such as a potentially unlimited amount of storage and computational capabilities. This new infrastructure exposes data to both internal and external threats as the institutions no longer have physical control of the infrastructure on which their data is stored as that is now in the control of the Cloud Service Providers (CSP).

One way of ensuring that user data is protected from both the CSPs and from unauthorized users is to encrypt the data before it is stored in the cloud using available encryption schemes. This grants only authorized users access to the required information (usually a key) to gain access to data in its original form. This system becomes a challenge when there is a need to provide fine-grained access to data to third parties in order to ensure that privacy of users are not violated in relation to privacy laws such as Personal Health Information Protection Act (PHIPA) \cite{pihipa} which governs the collection, use and disclosure of personal information in the health sector in the province of Ontario, Canada. Sharing of data with third parties such as universities and research institutes becomes a challenge as the health institutions need to ensure that privacy of their patients are not violated while they provide these parties with the data necessary for conducting research which could lead to advancements in the health industry and improve the service delivery for patients while saving more lives.

The need to meet the privacy and security requirements have led to the creation of systems for the sharing of electronic health data based on a system of encryption called Attribute Based Encryption (ABE) \cite{Ibraimi2010}\cite{Narayan2010}\cite{Akinyele2010}\cite{Barua2011}\cite{Alshehri}\cite{Hupperich2012}\cite{Hsieh2012}\cite{Li2013}. ABE, developed by Sahai and Waters \cite{Sahai2005}, provides the opportunity for fine-grained access control to data as either the user secret keys or the corresponding ciphertexts contain descriptive attributes with the other containing an access structure that specifies what attributes need to be present for decryption. This means that, if there is no match between the attributes of a user's key and the ciphertext, decryption is prevented. This enables the health institutions to be able to grant third parties access to only the specific data that they require to carry out their research while ensuring that Personal Identifiable Information (PII) of patients are not exposed in line with privacy laws and requirements.

\section{Research Challenge}

There is a need for systems developed to enable the secure exchange of Electronic Health Records to be based on more effective cryptographic schemes.  The challenge here is to use the latest developments in the area of ABE research to develop a system that while being more effective and robust than existing systems would offer similar levels of security and privacy with improvements in performance at the least. This would require the adaptation of an ABE scheme and its modification using existing techniques and placing it at the core of our developed system.

\section{Thesis Contributions}

In this thesis we have developed a framework for the sharing of health data among multiple parties even when the parties are using untrusted third party storage services. The platform employs symmetric encryption for the security of the actual data sets while a variant of the Multi-Authority Ciphertext Policy Attribute Based Encryption (MACP-ABE) is applied to encrypt the secret keys that users need to gain access to data in its plain form. This is because of the high amount of computation required for encryption and decryption algorithms of the different available Public Key encryption schemes, a category that includes the MACP-ABE scheme. The security of this protocol had been proven adequately in the original work \cite{Yang2014} and as a result, we have limited our analysis to the basic security requirements that are to be met by the framework.

The major contribution of this thesis is a framework for the exchange of electronic health data with significant decrease in the computational requirements in terms of time and resources.

\section{Thesis Outline}

The rest of this thesis is organized as follows: Chapter 2 provides some background technical information on related subjects such as security and privacy, electronic health data, cloud computing, cryptography and Attribute Based Encryption (ABE). It also provides a review of literature outlining work done in the area of ABE in order to improve the security and performance original schemes while adding features such as revocation. A review of systems that have been built based on ABE for the handling of health related data is also included. Chapter 3 introduces our secure privacy preserving framework for electronic health records. It provides the description and mathematical construction for the underlying ABE scheme and also a description of the architecture of our framework together with use cases. Chapter 4 provides details of the evaluation of our framework and also the results. It describes our experimental setup, provides a detailed performance evaluation with relation to computation and also provides a security analysis of the framework. Chapter 5 concludes the thesis, highlighting our contribution and also providing information on possible areas for future improvements.