\chapter{Introduction}
\label{chap:introduction}

%\chapter{Chapter Title Here} % Main chapter title
%\label{Chapter1} % For referencing the chapter elsewhere, use \ref{Chapter1} 

\section{Motivation}

Secure storage of electronic health data by health institutions, such as hospitals, is made challenging if it is to be made available to other parties while ensuring unauthorized access to user data is prevented and the privacy of their patients is protected. These institutions typically store this data on physical hardware in a secure location on their premises and so are able to prevent unauthorized access through the use of adequate network security infrastructure. They are also able to prevent authorized users from gaining access beyond their levels of control through the use of proper access control mechanisms which ensures accountability. The deployment of more efficient and modern information technology, such as cloud computing, extends the trust boundary of the Information Technology (IT) infrastructure of institutions beyond their facilities while giving them multiple advantages such as a potentially unlimited amount of storage and computational capabilities. However, this new infrastructure exposes data to both internal and external threats as the institutions no longer have physical control of the infrastructure on which their data is stored as that is now in the partial control of the Cloud Service Providers (CSP).

A health institution typically collects data of varying types which varies depending on the medical condition of their patients, with the patient name and id being the link between different data sets. Multiple research institutes may require access to only specific portions of this data across the multiple sets that are related to their area of specialization. This makes access control more complex depending on the number of users involved and their required level of access to data at a granular level. The health institution would want to be able to protect the privacy of their data from the CSP, while being able to grant their partners access to the appropriate portions of the database they have rights to as specified in their research partnership agreement. The access structure of the system becomes increasingly complex as multiple health institutions and research partners are added, with different individuals having similar or slightly varying levels of access across the platform.

%\textcolor{red}{I think an additional paragraph here would help with the motivation if you can describe one of scenarios we talked about, e.g. a researcher might be granted access to portions of multiple data sets at different institutions, which might overlap with other researchers access rights, etc.}

One way of ensuring that user data is protected from both the CSPs and from unauthorized users is to encrypt the data before it is stored in the cloud using available encryption schemes. This grants only authorized users access to the required information (usually a key) to gain access to data in its original form. This system becomes a challenge when there is a need to provide fine-grained access to data to third parties in order to ensure that the privacy of users is not violated in relation to privacy laws such as the Personal Health Information Protection Act (PHIPA) \cite{pihipa}. PHIPA governs the collection, use and disclosure of personal information in the health sector in the province of Ontario, Canada. Sharing data with third parties such as universities and research institutes becomes a challenge as the health institutions need to ensure patient privacy is not violated. The maintenance of patient privacy is especially important if data is to be shared for the purposes of necessary research which could lead to advancements in the health industry and improve the service delivery for patients while saving more lives.

The need to meet the privacy and security requirements have led to the creation of systems for the sharing of electronic health data based on a system of encryption called Attribute Based Encryption (ABE) \cite{Ibraimi2010, Narayan2010, Akinyele2010, Barua2011, Alshehri, Hupperich2012, Hsieh2012, Li2013}. ABE, developed by Sahai and Waters \cite{Sahai2005}, provides the opportunity for fine-grained access control to data. ABE achieves this be having an attribute based access policy attached to either the secret key of the user or the encrypted ciphertext. This depends on the variant of ABE because if the access policy is contained in the ciphertext then the corresponding attributes will have to be in the secret key and vice versa. This means that, if there is no match between the attributes of a user's key and the ciphertext, decryption is prevented. This enables the health institutions to be able to grant third parties access to only the specific data that they require to carry out their research while ensuring that Personal Identifiable Information (PII) of patients is not exposed in line with privacy laws and requirements.

\section{Research Challenge}

To allow for the kind of access required as described in the previous section, there is a need to develop a system that enables the secure exchange of Electronic Health Records. These systems would need to take advantage of existing advancements in the field of information technology, such as cloud computing, which gives individuals access to potentially unlimited amounts of storage and processing capabilities. These advancements further extend the trust boundaries of systems that use them as the physical devices that make up the infrastructure are not under the control of users. This makes the storage services that they provide potentially untrustworthy for use with highly sensitive data such as Electronic Health Records (EHRs). We also want to take advantage of the computational capabilities of these services for any operations that are potentially computationally intensive with regards to our data while maintaining the same levels of security and preserving the privacy of the information they contain.

To find a solution to this challenge we need to use effective cryptographic schemes that preserve privacy and provide the levels of security that we require. Security can easily be provided by existing cryptographic schemes but they are limited in their ability to provide fine grained access control and preserve privacy of information at a granular level. Using latest developments in the area of cryptographic research, specifically Attribute Based Encryption (ABE), takes us a step closer to finding a solution to the aforementioned challenge. The use existing techniques is also needed to modify an effective ABE scheme to be part of a developed framework so that we are able to take advantage of the computational abilities of the third party service with regards to the operations that are computationally intensive. This is with regards to most of the decryption operations of ABE schemes.

\section{Thesis Contributions}

In this thesis we have developed a framework for the sharing of electronic health data among multiple parties. Our framework ensures security and preserves privacy while allowing for the use of potentially untrusted third party services. The framework is based on a proposed Multi-Authority Ciphertext Policy Attribute Based Encryption with Outsourcing (MACP-ABEwO) scheme with securely outsourced decryption that is based on the ABE scheme in developed by Kan Yang and Xiaohua Jia\cite{Yang2014}. Our proposed MACP-ABEwO scheme provides effective decryption outsourcing in a multi-authority setting.

The platform employs symmetric encryption for the security of the actual data sets. The MACP-ABEwO scheme is applied to encrypt the secret keys that users need to gain access to data in its plain form. The use of symmetric encryption fo the actual is because of the high amount of computation required for encryption and decryption algorithms of the different available public key encryption schemes, a category that includes the MACP-ABEwO scheme. The security of this protocol had been proven thoroughly in the original work \cite{Yang2014} and as a result, we have limited our analysis to the basic security requirements that are to be met by the framework.

The major contribution of this thesis is a framework for the exchange of electronic health data with significant decrease in the computational requirements in terms of time and resources. This decrease is as a result of the outsourcing of the more computationally intensive operation typically involved in the ABE decryption process to the cloud. This enables the secure exchange of sensitive health data among health institutions and other third parties, such as research partners.


\section{Thesis Outline}

This thesis is organized as follows: Chapter 2 provides some background technical information on related subjects such as security and privacy, electronic health data, cloud computing, cryptography, and Attribute Based Encryption (ABE). It also provides a review of literature outlining work done in the area of ABE to improve the security and performance original schemes while adding features such as revocation. A review of systems that have been built based on ABE for the handling of health related data is also included. Chapter 3 introduces our secure privacy preserving framework for electronic health records. It provides the description and mathematical construction for the underlying ABE scheme and also a description of the architecture of our framework together with use cases. Chapter 4 provides details of the evaluation of our framework and also the results. It describes our experimental setup, provides a detailed performance evaluation with relation to computation and also provides a security analysis of the framework. Chapter 5 concludes the thesis, highlighting our contribution and also providing information on possible areas for future improvements.