\chapter{Evaluation and Results}
\label{chap:evaluation_and_results}

This chapter highlights the information about the software and hardware specifications involved in the implementation of the underlying framework and also shows the results of performance tests for the underlying ABE scheme in relation to scalability of computational overhead with regards to the number of attribute authorities and attributes involved in the system for data encryption, data decryption, and user revocation.

\section{Experimental Setup}

We evaluated the performance of our system framework by implementing the underlying ABE scheme in an environment using hardware and software tools as described below. We have focused on the performance of the underlying ABE scheme as there have been well established performance metrics for the multiple available types of symmetric encryption that could be used for the encryption of the actual data, with the ABE scheme used for the encryption of the symmetric key. 

\subsection{Hardware and Software Tools}
\begin{itemize}
	\item Linux Server - We used a Linux Server running Ubuntu 18.04.3 LTS with 12GB RAM size and an Intel Core i5-6400 CPU @ 2.70GH x 4 processor.
	
	\item Python - We wrote our underlying code using Python 3.7 as this gave us access to some other effective libraries that played a role in the robustness of our code and evaluation during our experiments.
	
	\item Charm Crypto Library\cite{charm13crypto} - We used the charm-crypto library for the implementation of the ABE components of our framework. Charm is a library implemented in python that has multiple contributors in the cryptographic field and as a result gives access to a robust tool set that provides functionalities related to pairing based public key encryption schemes such as ABE. Our scheme was implemented using 'SS512', a super-singular elliptic curve with a 512-bit base field with groups of prime order.
\end{itemize}

\section{Performance Evaluation}

We evaluated the performance of our system by evaluating the computational times for data encryption and decryption together with the time taken for users to update their attribute based secret keys. The focus of our evaluation are the processes that are run at the user end as we are working under the assumption that the Cloud Service Provider (CSP) provides in theory unlimited processing power unlike what could be available on the several devices at the user end.

We encrypt and decrypt sample symmetric keys using the underlying ABE scheme. We also use access policies that exclusively have 'AND' gates to push the limits of the maximum number of attributes that can be part of the access policy or user keys with respect to the number of Attribute Authorities (AA) involved. Our experiments have been run for 500 trials with the results being an average of the individual runs.

Our results for encryption as indicated in figure \ref{fig:res_enc_aas} show that our encryption system is identical in performance with the encryption algorithm of the adapted scheme for this framework \cite{Yang2014}. Encryption requires mostly exponential operations and the amount of time scales linearly with an increase in the number of AAs and attributes that are contained in the access policy under which the file is being encrypted.

\begin{figure}[]
	\centering
	\includegraphics[width=0.75\textwidth]{"Scheme Performance (Encryption_AAs)"}
	\caption{Encryption Computation Time}
	\label{fig:res_enc_aas}
\end{figure}

Our results for decryption show a constant decryption time irrespective of the number of attribute authorities or attributes in comparison with the increase in decryption time of the adapted scheme \cite{Yang2014}. This can be seen in figure \ref{fig:res_dec_aas}. These results indicate that the amount of time for decryption involving a single attribute will be the same as that involving a hundred or more attributes, keeping the computational overhead constant and allowing for the use of low computational devices for the decryption process. This is because decryption in the framework only requires a single exponentiation operation irrespective of the number of AAs or attributes. The pairing operations which would have been part of the decryption process as used in kan yang scheme\cite{Yang2014} have been outsourced to the Cloud Service Provider (CSP).

\begin{figure}[]
	\centering
	\includegraphics[width=0.75\textwidth]{"Scheme Performance (Decryption_AAs)"}
	\caption{Decryption Computation Time}
	\label{fig:res_dec_aas}
\end{figure}

Our setup for revocation was focused on showing how the time for revocation in terms of an existing user updating their attribute based secret keys when an existing user has been revoked scales with an increase in the number of attributes being revoked. Our results show that the time scales linearly with an increase in the number of attributes being revoked. This is as a result of the exponentiation operations that are involved in the update process. The result of our experiment can be seen in figure \ref{fig:res_rev_as}.

\begin{figure}[]
	\centering
	\includegraphics[width=0.75\textwidth]{"Scheme Performance (Revocation_As)"}
	\caption{Revocation Computation Time}
	\label{fig:res_rev_as}
\end{figure}

\section{Security Analysis}

In this section we analyze the different security and privacy properties of the framework. We describe how the framework is able to achieve the conditions necessary for confidentiality,integrity, and authentication including how it is able to preserve privacy through its processes. 

\subsection*{Confidentiality and Integrity}

The framework is able to ensure confidentiality by ensuring that users are only able to gain access to data that their access levels allow. Users are only able to decrypt data for which they have the required attributes that match the accompanying access structure. Also due to the randomness inserted in the generation of user secret key for attributes, multiple users are unable to combine their secret keys for individual attributes to gain access to data under a policy that they are individually unable to access. The integrity of data is ensured as users are only able to access data that they have been authorized for as there is a record of the different users who have been granted secret keys for attributes that match the policy.

The ability to revoke user access also enables the maintenance of the integrity of the system. Also the certificates generated by the Certificate Authority ensures that only valid users are able to request secret keys from the Attribute Authorities in the system. Also no single AA is able to grant itself access to data within the framework and the CA only plays the role of identity validation and does not have access to a universal key that grants it access to the stored encrypted data. Also the Cloud Service Provider (CSP) never has access to data in its plain form ensuring that confidentiality and integrity are maintained.

\subsection*{Authentication}

Our framework has authentication built into its cryptographic processes. The fact that users need access to certain attributes which they need to request from specific authorities is in its own way a form of access control. This ensures that not only do unauthorized users not gain any level of access to the system, the authorized users are only able to gain access to data that they have been authorized for, preventing them from being able to access data for which they do not have any permissions. The access policy acts as the rules while the attributes could be seen as the permissions that grant users access if they fit.

\subsection*{Privacy Preserving}

Privacy is preserved as part of our framework through the fact that data owners are able to have some control over the level of access to data by specifying the appropriate access policy when encrypting data before it is stored on the cloud. Also the ability of the system to revoke access also ensures that user access can have a time limit depending on the preset rules before they are granted access. Also the strong underlying confidentiality, integrity and authentication mechanisms aid in ensuring data privacy.