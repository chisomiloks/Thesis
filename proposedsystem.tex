\chapter{Proposed System}
\label{chap:proposedsystem}


\section{Summary of Proposed System Currently}

The proposed system displayed in figure Figure \ref{systemarchitecture} will have five main entities similar to \cite{Yang2013}\cite{Yang2014} which include:

\begin{figure}
\begin{center}
\includegraphics[scale=0.45]{system_diagram.png} 
\caption{System Architecture Diagram}
\label{systemarchitecture}
\end{center}
\end{figure}

\begin{enumerate}

\item Data Creators (DC) - The data owners are the generators of data to be stored in the cloud and would have control of what requirements need to be met in order for any other entity to have access to the data stored in the server. The use of Ciphertext Policy Attribute Based Encryption (CP-ABE) gives the users this control as they specify the access policy that needs to be satisfied for decryption to be possible. There are multiple data owners in the proposed system which would typically be the hospitals where the Electronic Medical Records (EMR) are being generated.

\item Data Users (DU) - The data users in the system are those entities that require access to the data generated and stored by the several data owners in the system. The required level of access is determined by whatever agreements are made and are also enforced by the attributes that the users are able to gain from the appropriate authorities to enable their access. The data users could include other data owners in the system and other entities such as researchers who require access to the EMR stored on the servers.

\item Cloud Server - The cloud server provides an avenue for potentially unlimited data storage space for the EMRs and also as a system for executing most of the computational intensive processes that are part of the underlying system. The cloud server is assumed to be semi trusted and data is never exposed in plain format on the cloud at any point in time as this would be a compromise of the security and privacy requirements of the system.

\item Attribute Authorities (AA) - There are multiple AAs that make up the system. The AAs are responsible for managing several disjoint set of attributes which collectively make up the attribute universe of the entire system. The AAs generate a public/secret key pair for the different attributes that they control. With the public keys for the AAs, data owners are able to encrypt data specifying certain attributes that are part of the access policy and under the control of the various involved AAs, thereby ensuring that only users who have been able to acquire the appropriate secret keys for the corresponding attributes from the AAs can gain access to the data.

\item Certificate Authority (CA) - This is an entity that comes with the underlying MA-ABE architecture which forms the foundation of the system\cite{Yang2013}\cite{Yang2014}. The CA is responsible for the registration of users and AAs, and serves as a means for the verification of the different entities in the system as they interact with each other. The CA is in no way directly part of the process of attribute or secret key generation and has no access to either the attributes, keys or the data in the system. The CA assigns globally unique identity numbers to the various other entities as well as generates additional information which is only used for verification purposes.

\end{enumerate}


\subsection{Sample Use Case Scenario}

Hospital A in a particular province registers with the CA and acquires a global ID and other supporting information for identity verification. Hospital A is under different governmental and research ethic boards which enforce the different privacy and security laws that guide the use of EMRs. Hospital A wants to grant University B's Health Informatics Research (HIR) team made up of faculty from both the Health Sciences and Electrical and Computing Engineering departments access to a collection of ECG signals for a certain subset of patients. Hospital A encrypts all the ECG data for the patients in their EMR records using an individual symmetric key that is provably secure such as an AES key. Hospital A then encrypts this symmetric key using ABE with an access policy that says \textit{Staff \textbf{OR} (Hospital A Researcher \textbf{AND} (University B \textbf{AND} (Health Sciences Faculty \textbf{OR} Electrical and Computer Engineering Faculty) \textbf{AND} HIR Team))} through the use of the public keys gotten from the authorities in charge of staff at the hospital, the research ethics board at the hospital responsible for registering research partners, the research ethics board at the university responsible for approving research groups in the university, and the university authorities.

As a result of the access policy, only users who are able to acquire the appropriate attributes, that is their roles match those stated in the access policy, are able to gain access. An individual who is part of the HIR team, and also a member of faculty in either Health Sciences or Electrical and Computer Engineering but not a researcher at Hospital A, would not be able to decrypt the encrypted data. A user who is a researcher in Hospital B will not be able to collude with the earlier user by combining their individual keys to gain access as a result of embedded unique random entities that are tied to the unique identities of each user.

In order to revoke a particular user's access based on a particular attribute or a group of attributes, the components of the secret keys of other legitimate users are updated and the corresponding components in the ciphertexts are also updated, thereby preventing a revoked user from gaining access to future encrypted data and allowing new users to have access to both new and old data in the system. The update of components for the ciphertexts are executed using proxy re-encryption. This prevents the cloud server from having access at any point to the data in plain form.
