%----------------------------------------------------------------------------------------
%	CUSTOM STUFF
%----------------------------------------------------------------------------------------
%%%%%%%%%%%%%%%%%%%%MATH AND CHEM PACKAGES%%%%%%%%%%%%%%%%%%%%%%%%%%%%%%%%%%
\usepackage{amsmath} %mathy things
\usepackage{siunitx} %all of the pretty units with correct spacing. To be use with \SI{number}{\unit} or \SI{number}{unit} e.g. \SI{5}{\degree} or \SI{5}{mm}
\usepackage[version=3]{mhchem}

% for maths symbols
\usepackage{amssymb}
\usepackage{bbold}

% for unnumbered theorem environs
\usepackage{amsthm}

%%%%%%%%%%%%%%%%%%%CONTROLLING FLOATS%%%%%%%%%%%%%%%%%%%%%%%%%%%%%%%%%%%%%%%
%controlling floats: HTB with higher importance than htb and \Floatbarrier
\usepackage{placeins}
\usepackage{float}

%%%%%%%%%%%%%%%%%%%%COLOURS%%%%%%%%%%%%%%%%%%%%%%%%%%%%%%%%%%%%%%%%%%%%%%%%%
%colour packages loaded by default
%use \textcolo{option}{your text goes here}
%defining specific colours
%\definecolor{gray}{rgb}{0.5,0.5,0.5}
%\definecolor{mauve}{rgb}{0.58,0,0.82}

%%%%%%%%%%%%%%%%%%%%STICKY NOTES%%%%%%%%%%%%%%%%%%%%%%%%%%%%%%%%%%
%use with \task{} \fix{} and \remark{}. \todo{} will appear yellow
\usepackage[obeyDraft]{todonotes}
\usepackage{xifthen}

\newcommand{\remark}[2][]{
\ifthenelse{\isempty{#1}}
{\todo[color=green!40]{#2}}
{\todo[color=green!40,#1]{#2}}
}
\newcommand{\task}[2][]{
\ifthenelse{\isempty{#1}}
{\todo[color=yellow!70]{#2}}
{\todo[color=yellow!70,#1]{#2}}
}
\newcommand{\fix}[2][]{
\ifthenelse{\isempty{#1}}
{\todo[color=red!70]{#2}}
{\todo[color=red!70,#1]{#2}}
} 

%%%%%%%%%%%%%%%%%%%INCLUDING CODE%%%%%%%%%%%%%%%%%%%%%%%%%%%%%%%
\usepackage{listings}
%lstset allows coloured code hunks like python or CC with thier proper syntax highlighting. Should be used instead of a verbatim section.
\lstset{frame=tb,
  %language=Matlab,
  %language=Python
  %language=C
  %language=Java
  aboveskip=3mm,
  belowskip=3mm,
  showstringspaces=false,
  columns=flexible,
  basicstyle={\small\ttfamily},
  numbers=none,
  numberstyle=\tiny\color{gray},
  keywordstyle=\color{blue},
  commentstyle=\color{dkgreen},
  stringstyle=\color{mauve},
  breaklines=true,
  breakatwhitespace=true,
  tabsize=2
}
%\lstset{language=Python} %%Set the language of lstset before your code piece. 


% Adjust enumerate numbering or lettering
\usepackage{enumitem}

% definitions & examples
\theoremstyle{definition}
\newtheorem{definition}{Definition}[section]
\newtheorem{example}{Example}[section]

% remarks
\theoremstyle{remark}

% theorems, corollaries & lemmas
\newtheorem{theorem}{Theorem}[section]
\newtheorem{corollary}{Corollary}[theorem]
\newtheorem{lemma}[theorem]{Lemma}


\usepackage{listings}

% Adjust the appearance of the Table of Contents
\usepackage{tocbibind}
%\usepackage[titles,subfigure]{tocloft}