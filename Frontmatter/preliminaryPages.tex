%----------------------------------------------------------------------------------------
%	CERTIFICATE OF APPROVAL
%----------------------------------------------------------------------------------------

%put a tilte:
%title:Thesis examination information
%Subtitle: Submitted by: Name

%Degree name in Program name

%\noindent
%{\Large \ttitle} \newline
%Submitted by \authorname \newline

%\noindent
%\degreename \hspace{0.07cm} in \subjectname

%\vspace{1cm}

%An oral defense of this thesis took place on October 19, 2018 in front of the following examining committee: 

%      \begin{table}[h!]
%      \centering
      %\setlength{\tabcolsep}{0.125cm}
%      \caption*{Examining Committee}
        %\hspace*{-4.0cm}
%         \scalebox{1.0}{
%        \begin{tabular}{@{}ll@{}}
%          \toprule
%          Chair of Examining Committee     & Dr. Big Bird       \\
%          Research Supervisor    &  Dr. Elmo     \\      
%          Examining Committee Member & Dr. The Count \\
%          Examining Committee Member & Dr. Oscar\\ 
%          Thesis examiner & Dr. Bert, (Masters), affiliation \\
%          External Examiner & Dr. Bert, affiliation \\
%          \bottomrule
%        \end{tabular}
%        }
%      \end{table}


%The above committee determined that the thesis is acceptable in form and content and that a satisfactory knowledge of the field covered by the thesis was demonstrated by the candidate during an oral examination.  A signed copy of the Certificate of Approval is available from the School of Graduate and Postdoctoral Studies.

\cleardoublepage

%----------------------------------------------------------------------------------------
%	ABSTRACT PAGE
%----------------------------------------------------------------------------------------
\begin{abstract}
\addchaptertocentry{\abstractname} % Add the abstract to the table of contents

There exists a need for sharing user health data, especially with institutes for research purposes. The need to protect this user data from malicious and unauthorized access continues to present itself. This is especially true in the case of a system that includes a third party storage service, which limits the amount of control that the data owner has. The use of encryption for secure data storage has continued to evolve in order to meet the need for flexible and fine grained access control. This evolution has led to the development of Attribute Based Encryption (ABE) based on the concept of Identity Based Encryption (IBE), an extension of public key cryptography. ABE gives more control to the data owner and its application enables users make use of the technological advancements available to them.

In the area of health, there is a lot of data collected on a constant basis. This medical data is highly sensitive as it typically contains the personal information of the users together with other sensitive information that needs to remain private. This data is to be protected from unauthorized users while ensuring it is available in a controlled form to those who require access in order to provide care. There is also a desire to share this data with other medical researchers and third parties such as universities and research institutes. This opens an avenue for these third parties to be able to carry out research using this data in order to contribute to the advancement of the field of health care. There is also a need to protect user privacy by ensuring that third parties only have access to the limited amount of data they require for their research while protecting the Personal Identifiable Information (PII) of users. There is also a need to follow the guidelines set by the laws with regards to user data privacy.

Advancements in the area of technology such as cloud computing has expanded the trust boundaries of user systems while providing multiple advantages such as potentially unlimited access to storage and computational resources. In order to take advantage of these resources while maintaining the same levels of security and privacy, mechanisms need to be deployed. This is because users do not have physical control of the devices that are providing this resources and so need to be able to protect their data from a potentially compromised system administrator (i.e the Cloud Service Provider). There is a need to find a balance between providing adequate levels of security and finding a way to provide effective fine-grained access control to potential users. ABE provides an effective mechanism that fits this requirement as its use of attributes to regulate access provides embedded access control together with the security provided by it cryptograhic system.

The use of ABE to ensure the security and privacy of health data has been explored with multiple solutions proposed. This thesis presents a framework and to apply within it an improved ABE scheme which allows for the secure outsourcing of the more computationally intensive processes for data decryption to the cloud servers. The aims of this improved ABE scheme are to reduce the amount of time needed for decryption to occur at the user end and also reduce the amount of computational power needed by users who require access to the data stored in the cloud in its encrypted form.

%The Thesis Abstract is written here (and usually kept to just this page). The page is kept centered vertically!

%The abstract occupies a single page and provides a summary of the thesis outlining the problem, methods of investigation, main results and general conclusions. The abstract must give enough information about the thesis to allow a potential reader to decide whether or not to consult the complete work. It should not include graphs, charts, illustrations, tables or references.

%The abstract of a master’s thesis must not exceed 150 words, while the abstract of a doctoral thesis may be up to 350 words. These guidelines will be strictly enforced, or so they say over at SGPS...

%At the end of the Abstract should appear a list of keywords. 
%\\[0.5cm]
%\textbf{Keywords:} \LaTeX, UOIT, Thesis
\end{abstract}

%----------------------------------------------------------------------------------------
%	DECLARATION PAGE
%----------------------------------------------------------------------------------------
%\begin{declaration}
%\addchaptertocentry{\authorshipname} % Add the declaration to the table of contents
%\noindent I, \authorname, declare that this thesis titled, \enquote{\ttitle} and the work presented in it are my own. I confirm that:

%\begin{itemize} 
%\item I hereby declare that this thesis consists of original work of which I have authored. This is a true copy of the thesis, including any required final revisions, as accepted by my examiners.
%\item I authorize the University of Ontario Institute of Technology to lend this thesis to other institutions or individuals for the purpose of scholarly research. I further authorize University of Ontario Institute of Technology to reproduce this thesis by photocopying or by other means, in total or in part, at the request of other institutions or individuals for the purpose of scholarly research. I understand that my thesis will be made electronically available to the public.\\
%\end{itemize}

%If your research need ethics approval this goes here. 
 
%\noindent Signed:\\
%\rule[0.5em]{25em}{0.5pt} % This prints a line for the signature
 
%\noindent Date:\\
%\rule[0.5em]{25em}{0.5pt} % This prints a line to write the date
%\end{declaration}




%%----------------------------------------------------------------------------------------
%%	STATEMENT OF CONTRIBUTIONS
%%----------------------------------------------------------------------------------------
%
%\begin{contributions}
%\addchaptertocentry{\contributionsname} 

%This section must include any contributing people or funding for the Thesis, talk with supervisor to ensure you have the correct statements. If you have published the work IT MUST be noted here, if you have not there is a sentence you can copy+paste into here from UOIT template.

%From the Guidelines: In the case where a thesis includes papers co-authored by the candidate and others, the thesis must state explicitly who contributed to such work and the nature and extent of this contribution. The Supervisor(s) must attest to the accuracy of such statements about co-authorship at the oral examination.

%\end{contributions}

%----------------------------------------------------------------------------------------
%	DEDICATION
%----------------------------------------------------------------------------------------

\dedicatory{For/Dedicated to/To \ldots my imaginary friends.} 

\cleardoublepage
%----------------------------------------------------------------------------------------
%	QUOTATION PAGE
%----------------------------------------------------------------------------------------

%\vspace*{0.2\textheight}

%\noindent\enquote{\itshape 'You can't give her that!' she screamed. 'It's not safe!'\\ IT'S A SWORD, said the Hogfather. THEY'RE NOT MEANT TO BE SAFE.\\
%'She's a child!' shouted Crumley.\\
%IT'S EDUCATIONAL.\\
%'What if she cuts herself?'\\
%THAT WILL BE AN IMPORTANT LESSON..}\bigbreak
%
%\hfill Terry Pratchett

\cleardoublepage


%----------------------------------------------------------------------------------------
%	ACKNOWLEDGEMENTS
%----------------------------------------------------------------------------------------

\begin{acknowledgements}
\addchaptertocentry{\acknowledgementname} % Add the acknowledgements to the table of contents

- Acknowledge supervisor

- External academic support or guides

- External professional support (slc, ogs staff)

- Colleagues (lab mates, lokendra,etc)

- Personal Support (Liz, Tim, Almey, Tessa, Catherine, Margot, Aida, etc)

- Family (Mum, Dad, Brothers, etc)

%From the guidelines: This is a brief acknowledgment of assistance given to the candidate in his/her research and writing. The content and format of this page are up to the student.
\end{acknowledgements}