\chapter{Secure Privacy Preserving Framework for Electronic Health Records (EHRs)}
\label{chap:proposed_solution}


\renewcommand{\theenumi}{\Alph{enumi}}
\renewcommand{\theenumii}{\roman{enumii}}


\subsection*{Chapter ToDo}
\textcolor{red}{
\begin{itemize}
	\item Overview	
	\begin{itemize}
		\item what the subsequent sections will contain to provide a clearer image of their content
		\item what are the good things about the model: its more secure than xyz, runs faster. potential drawbacks: ... its own section?
		\item what the chapter will contain: description of the proposed framework for secure storage of data using a combination of attribute and symmetric-based encryption.  details of implementation. Also provide info on the security model for the underlying attribute based encryption scheme. This would highlight the assumptions that was used to verifying the integrity of the proposed scheme. I will summarize the pros and cons of the proposed scheme.
	\end{itemize}
	\item ABE Scheme Construction
	\begin{itemize}
		\item Decryption Phase 1 Construction Section - Add more info about how secret shares are generated [more info can be found in papers by brent water or allison lewko for shorter forms....long form take a look at the PhD thesis of ben lynn]
		\item Decryption Phase 2 Construction Section - Possible rephrasing of statement before formula [look to replace the term binding element or whole sentence]
		\item Revocation Construction Section - possibility of more info to the description before the formulas
	\end{itemize}
	\item Look up impact of security parameter...has to do with key length and curve size and stuff like that most likely \textcolor{black}{(security parameter plays a role in determine the group size..which affects the key size and has an impact on the security of the scheme....no impact on correctness of the scheme though.)}
	\item Determine if Security Model/Security Game + Assumptions are to be included between the scheme definition and construction chapters
	\begin{itemize}
		\item Look up info from adopted paper + look into security in outsourcing ABE papers also
		\item Crypto Hardness problems that apply here
		\item Security Game
		\item What assumption(s) is from the point of view of security is the system to be analyzed by
		\item Correctness/Completeness (same thing) and other high level proof methods
	\end{itemize}
	\item System Process Diagram - the system process is to provide a description of the interactions between the multiple entities from a high level for the system
	\item Use Case
	\begin{itemize}
		\item Use real life example to show what the entities could be in the real world
		\item Individuals can also play the role of DU and DO just with different amounts of access control
	\end{itemize}
\end{itemize}
}



\section{Overview}

This chapter introduces the framework that is the basis of this thesis. We start with a description of the ABE scheme that plays an essential role in providing the features of security and privacy for the framework. The construction of the ABE scheme follows after the definition of the various algorithms. The construction shows the mathematical concepts that confirm the inner workings of the different algorithms.

A description of the entire framework follows with the architecture and its corresponding entities highlighted together with a detailed use case that describes how framework can be implemented in a real world scenario.


\section{Multiple Authority Ciphertext Policy Attribute Based Encryption with Outsourced Decryption}

\subsection{ABE Scheme Definition}

Our ABE scheme has been adapted from the scheme developed in \cite{Yang2014}. Using similar methods applied in \cite{Green2011outsource}, we have modified the scheme with the aim of outsourcing the bulk of the decryption algorithm operation to the CSP. This section contains a definition of the algorithms that make up our ABE scheme.

\subsubsection*{Setup Algorithms}
The setup algorithms are run as a part of the initialization stage of the ABE scheme. These algorithms are responsible for the registration of the other entities in the system. This also provides them with the necessary information for verifying their identity when communicating with other entities in the system and other vital information they need in order to execute their own operations
\begin{itemize}
	\item CASetup - The CASetup algorithm is the algorithm executed by the Certificate Authority in order to generate the Global Master Key (GMK) and Global Public Parameters (GPP) of the entire scheme. It takes as input the security parameter of the scheme. The CASetup algorithm also takes care of the registration of the users and attribute authorities (AAs) in the system by assigning to them their unique ids. It is also responsible for generating the public-secret key pairs together with the user certificates and their corresponding verification keys. It makes the GPP available to all registered entities in the system and sends an unmatched public-private key pair for all users to the AA.
	
	\item AASetup - The AASetup algorithm is run by the individual Attribute Authorities (AAs) and generates the public-secret key pair for the attribute authority. It also generates a public-version key pair for each of the attributes under the control of the AA that is provided as input in the AA universal attribute set.
\end{itemize}

\subsubsection*{Key Generation Algorithm}
\begin{itemize}
	\item SKGen - The SKGen algorithm is run by the corresponding AAs in order to provide the user with the secret key for the attributes that they have been assigned. The AA takes as input the system GPP, the users Global Public Keys (GPKs), one of the user GSKs, the set of attributes and their corresponding public-version key pairs. It provides the user with a secret key that can be used for decryption.
\end{itemize}

\subsubsection*{Encryption Algorithm}
\begin{itemize}
	\item Encrypt - The Encrypt algorithm is run by the data owner in order to encrypt the corresponding piece of data. This algorithm takes as input the system GPP, the public keys (PKs) for the AAs involved in the encryption - as they are in control of the attributes that make up the policy under which data is to be encrypted - of the data, and an access policy. It provides as output a ciphertext (CT) which implicitly contains the corresponding access policy.
\end{itemize}

\subsubsection*{Decryption Algorithms}
\begin{itemize}
	\item CTransform - The CTranform algorithm is responsible for the transformation of the CT into an El Gamal style ciphertext that can be easily decrypted by the end user. This algorithm is executed by the Cloud Service Provider (CSP) and takes as input the ciphertext, the secret keys of the attributes in the user's possession and the user's GPK. It produces as output a partially decrypted ciphertext (CT') is the user's secret keys meet the policy under which the data was encrypted, otherwise it outputs an error.
	
	\item Decrypt - The Decrypt algorithm is run by the end user and takes as input the partially decrypted ciphertext CT' and the user's global secret key which is not shared with any other entity in the system and produces the original data as output for the user in its plain form.
\end{itemize}

\subsubsection*{Revocation Algorithms}
\begin{itemize}
	\item UKeyGen - The UKeyGen algorithm is run by the AA under whose control the revoked attribute falls. The algorithm takes as input the SK of the AA, the revoked attribute and the current version key for the revoked attribute. It produces as output an updated version key for the revoked attribute. It produces an update key to be used to update the corresponding secret keys of users who possess the revoked attributes and whose access is not being revoked. It also produces an update key for the affected ciphertexts whose access policies contain the revoked attribute to allow for users who still maintain access to be able to decrypt while denying revoked users further access.
	
	\item SKUpdate - The SKUpdate algorithm is run by each non revoked user in the system and takes as input the corresponding secret key acquired from the AA that control the affected attribute together with the update key for user secret keys generated by the UKeyGen algorithm. It outputs a new secret key for the user.
	
	\item CTUpdate - The CTUpdate algorithm is run the CSP and takes as input the affected ciphertexts and the update key generated by the UKeyGen algorithm for ciphertext update. it outputs a new ciphertext that can only be decrypted by users with a current version of the corresponding revoked attribute.
	
\end{itemize}

%\subsubsection{Security Model}

\subsection{ABE Scheme Construction}


\begin{enumerate}
	
	\item CA Setup - This procedure is run by the CA to setup the system $$ CASetup(1^{\lambda}) \longrightarrow GMK, GPP $$
	where $ 1^{\lambda} $ is the security parameter, $ GMK $ is the Global Master Key, and $ GPP $ represents the Global Public Parameters.
	
	The CA chooses two multiplicative groups $ G, G_{T} $ with the same prime order $ p $ and a bilinear map $ e \colon G \times G \rightarrow G_{T} $ and a hash function $ H \colon \{0, 1\} \rightarrow G. $ The CA chooses two random numbers $ a, b \in Z_{p} $
	
	The GMK is set as $ (a, b) $ and the GPP is set to $ (g, g^{a}, g^{b}, H) $
	
	\begin{enumerate}
		
		\item User Registration
		This is executed for all users in the system by the CA. Each user is assigned a globally unique $ uid $ and for each $ uid $, the CA generates two random numbers $ u_{uid}, u'_{uid} \in Z_{p} $ as its global secret keys
		\begin{align*}
			GSK_{uid} = u_{uid}, GSK'_{uid} = u'_{uid}
		\end{align*}
		The global public keys for each user is generated as 
		\begin{align*}
			GPK_{uid} = g^{u_{uid}}, GPK'_{uid} = g^{\frac{1}{u'_{uid}}}
		\end{align*}
		The CA generates a $ Certificate{uid} $ for each user $ uid $ and send $ (GPK_{uid}, GSK'_{uid}, \\ Cerificate(uid)) $ to the user.
		
		\item Attribute Authority (AA) Registration
		The CA assigns a globally unique authority identity $ aid $ to each AA and sends $ (GPK'_{uid}, GSK_{uid}) $ for all registered users to $ AA_{aid} $. The CA also sends the verification key $ vk_{CA} $ to $ AA_{aid} $ for verifying the $ Certificate(uid) $ assigned to each user.
		
	\end{enumerate}
	
	\item AA Setup
	Let $ X_{aid} $ be the set of all attributes managed by $ AA_{aid} $. The AA chooses three random numbers $ \alpha_{aid}, \beta_{aid}, \gamma_{aid} \in Z_{p} $ as its secret key.
	\begin{align*}
		SK_{aid} = (\alpha_{aid}, \beta_{aid}, \gamma_{aid}) \\
		PK_{aid} = (e(g, g)^{\alpha_{aid}}, g^{\beta_{aid}}, g^{\frac{1}{\beta_{aid}}})
	\end{align*}
	For each attribute $ x_{aid} \in X_{aid} $, $ AA_{aid} $ generates a public attribute key as $ PK_{x_{aid}} = (PK_{1, x_{aid}} = H(x_{aid})^{v_{x_{aid}}}, PK_{2, x_{aid}} = H(x_{aid})^{v_{x_{aid}}\gamma_{aid}}) $ where $ v_{x_{aid}} $ is the version key of attribute $ x_{aid} $ i.e $ VK_{x_{aid}} = v_{x_{aid}} $.
	
	\item Secret Key Generation
	The Attribute Authority with $ AA_{aid} $ assigns a set of attributes $ S_{uid,aid} $ to user with $ uid $ after authentication and certificate verification. The AA chooses a random number $ t_{uid,aid} \in Z_{p} $ and computes a secret key for the user as
	\begin{align*}
		SK_{uid,aid} = (K_{uid,aid} = g^{\frac{\alpha_{aid}}{u'_{uid}}}g^{au_{uid}}g^{bt_{uid,aid}}, K'_{uid,aid} = g^{t_{uid,aid}}, \\
		\forall x_{aid} \in S_{uid,aid} \colon K_{x_{aid},uid} = g^{t_{uid,aid}\beta_{aid}}H(x_{aid})^{v_{x_{aid}}\beta_{aid}(u_{uid} + \gamma_{aid})})
	\end{align*}
	If the user $ uid $ does not hold any attributes from $ AA_{aid} $, the user secret key $ SK_{uid,aid} $ only contains $ K_{uid,aid} $.
	
	\item Encryption
	Data is divided into components based on the level of granularity required for access control i.e $ m = \{m_{i}, \cdots, m_{n}\} $. Data components are encrypted using symmetric encryption keys $ \kappa = \{\kappa_{i}, \cdots, \kappa_{n}\} $. An access structure $ (M_{k}, \rho) $ is defined for each content key $ \kappa_{i} (i = 1, \cdots, n) $ and encrypted using the ABE scheme to produce the corresponding ciphertext. Let M be an $\ell$ x n matrix, where $\ell$ denotes the total number of all the attributes and the function $ \rho $ associates rows of M to attributes. The function $\rho$ is not required to be injective which allows for an attribute to be associated with more than one row of M.
	
	To encrypt the content key $\kappa_{i}$, the algorithm chooses a random element $ s \in Z_{p} $ which is used as the random encryption exponent. It then selects a random vector $ \vec{v} = (s, y_{2},\ldots,y_{n}) \in Z_{p} $ where $ y_{2},\ldots,y_{n} $ are used to share the encryption exponent s. It then computes $ \forall 1 \leq i \leq \ell: \lambda_{i} = \vec{v}.M_{i} $ where $ M_{i} $ is the vector corresponding to the i-th row of M. It then randomly selects $ r_{1},r_{2},\ldots,r_{\ell} $ and computes the ciphertext as	
	\begin{align*}
		&CT_{K_{i}} = \biggl(C = K_{i}\cdot\Bigl(\prod\limits_{aid_{k} \in I_{A}}PK_{aid_{k}}\Bigl)^{s}, C' = g^{s}, C'' = g^{bs},\\
		&\forall 1 \leq i \leq \ell, \rho(i) \in X_{aid_{k}} \colon C_{i} = g^{a\lambda_{i}\cdot\bigl(PK_{i,\rho(i)}\bigl)^{-r_{i}}}, C'_{i} = g^{r_{i}},D_{i} = g^{\frac{r_{i}}{\beta_{aid_k}}}, D'_{i} = \Bigl(PK_{2,\rho_{(i)}}\Bigl)^{r_{i}}\biggl)
	\end{align*}
	Then the encrypted data is uploaded to the cloud server by the owner.
	
	\item Decryption
	\begin{enumerate}
		
		\item Phase 1 - Ciphertext Transformation (Partial Decryption)\\
		This is the first phase of the data encryption component that involves the transformation of the ciphertext into an El Gamal style ciphertext referred to in this framework as a token through partial decryption while the integrity of the original message is preserved.
		
		\begin{align*}
			CT' = \prod\limits_{aid_{k} \in I_{A}} \frac{e\Bigl(C', K_{uid,aid}\Bigl)e\Bigl(C'', K'_{uid,aid}\Bigl)^{-1}}{\prod\limits_{i \in I_{aid_k}}\biggl(e\Bigl(C_i, GPK_{uid}\Bigl)e\Bigl(D_i, K_{\rho(i),uid}\Bigl)e\Bigl(C'_i, K_{uid,aid_k}^{-1}\Bigl)e\Bigl(g, D'_i\Bigl)^{-1}\biggl)^{\omega_{i}n_A}}
		\end{align*}
		
		$\omega_{i}$ - constant chosen for the reconstruction of encryption exponent s using shares $\lambda_{i}$
		
		$n_A$ - number of AAs involved in the ciphertext
		
		$\rho(i)$ - mapping of each row of access structure to attribute i
		
		\begin{align*}
			CT' = \prod\limits_{aid_{k} \in I_{A}} e\Bigl(g, g\Bigl)^{\frac{s\alpha_{aid}}{u'_{uid}}}
		\end{align*}
		
		
		\item Phase 2 - Data Decryption
		The user does an exponentiation on the token to get the blinding element (BE) for decryption.
		
		\begin{align*}
			BE= CT'^{u'_{uid}} = \prod\limits_{aid_{k} \in I_{A}} e\Bigl(g, g\Bigl)^{s\alpha_{aid}}
		\end{align*}
		
		Remember the $C$ element of the ciphertext $= K_i\cdot\bigl(\prod\limits_{aid_{k} \in I_{A}}PK_{aid_{k}}\bigl)^{s}$ where $PK_{aid_{k}} = e\bigl(g, g\bigl)^{\alpha_{aid}}$. Therefore the original message which in this case is the symmetric key is computed as
		
		\begin{align*}
			K = \frac{C}{BE}
		\end{align*}
	\end{enumerate}
	
	\item User Revocation
	User revocation which can also be referred to as attribute revocation as used in some of the literature on the subject should achieve two critical criteria. The revoked user should not be able to decrypt new ciphertexts which have been encrypted using the public attribute keys of attributes that the user was previously granted private keys for. This is referred to as Backward Security. Any new user who has the right attributes should be able to decrypt the original ciphertexts that were encrypted using those public parameters. This is referred to a Forward Security. The revocation algorithm used has been gotten from the scheme by Yang et. al in \cite{Yang2014}.
	
	
	\begin{enumerate}
		\item Update Key Generation
		In order to revoke an attribute $\tilde{x}_{aid'}$ from a particular user, the attribute authority (AA) responsible $AA_{aid'}$ runs the key update algorithm taking as input the secret key $SK_{aid'}$ of the 
		attribute authority, the revoked attribute $\tilde{x}_{aid'}$, and its current version key $VK_{\tilde{x}_{aid'}}$.
		The algorithm generates a new version key $VK'_{\tilde{x}_{aid'}} = v'_{\tilde{x}_{aid'}} (v'_{\tilde{x}_{aid'}} \neq v_{\tilde{x}_{aid'}})$ for the revoked attribute $\tilde{x}_{ai'd}$.
		
		The $AA_{aid'}$ then generates a unique update key ${UK}_{s,\tilde{x}_{aid'},uid}$ for secret key update by each non-revoked user $uid$ as
		\begin{align*}
			{UK}_{s,\tilde{x}_{aid'},uid} = H(\tilde{x}_{aid'})^{\beta_{aid'}(v'_{\tilde{x}_{aid'}}-v_{\tilde{x}_{aid'}})(u_{uid}+\gamma_{aid'})}
		\end{align*}
		and generates the update key ${UK}_{c,\tilde{x}_{aid'},uid}$ for ciphertext update as
		\begin{align*}
			{UK}_{c,\tilde{x}_{aid'},uid} =\Bigl( UK_{1,\tilde{x}_{aid'}} = \frac{v'_{\tilde{x}_{aid'}}}{v_{\tilde{x}_{aid'}}}, UK_{2,\tilde{x}_{aid'}} = \frac{v_{\tilde{x}_{aid'}} - v'_{\tilde{x}_{aid'}}}{v_{\tilde{x}_{aid'}}\gamma_{aid'}} \Bigl)
		\end{align*}
		The $AA_{aid'}$ sends the ${UK}_{s,\tilde{x}_{aid'},uid}$ to the non-revoked user $uid$ and sends ${UK}_{c,\tilde{x}_{aid'},uid}$ to the cloud server.
		
		The $AA_{aid'}$ then updates the public attribute key of the revoked attribute $\tilde{x}_{aid'}$ as
		\begin{align*}
			\widetilde{PK}_{\tilde{x}_{aid'}} = ({PK}_{\tilde{x}_{aid'}})^{UK_{1,\tilde{x}_{aid'}}}
		\end{align*}
		
		
		\item Secret Key Update
		This process is run by the non revoked users in the system. The algorithm produces an updated secret key using the update key ${UK}_{s,\tilde{x}_{aid'},uid}$ for the individual users as
		\begin{align*}
			\widetilde{SK}_{uid,aid'} = \Bigl(&\tilde{K}_{uid,aid'} = {K}_{uid,aid'},\tilde{K'}_{uid,aid'} = {K'}_{uid,aid'},\\
			&\tilde{K}_{\tilde{x}_{aid'},uid} = {K}_{\tilde{x}_{aid'},uid}.{UK}_{s,\tilde{x}_{aid'},uid},\\
			&\forall x_{aid'} \in S_{uid,aid'} \backslash \{ \tilde{x}_{aid'} \} \colon \tilde{K}_{\tilde{x}_{aid'},uid} = {K}_{\tilde{x}_{aid'},uid}  \Bigl) 
		\end{align*}
		
		\item Ciphertext Update
		\begin{align*}
			\widetilde{CT} = \Bigl(&\tilde{C} = C, \tilde{C'} = C', \tilde{C''} = C'',\\
			&\forall 1 \leq i \leq \ell: \tilde{C'}_{i} = C'_{i}, \tilde{D}_{i} = D_{i},\\
			&if \rho(i) = \tilde{x}_{aid'} : \tilde{C'}_{i} = C_{i} . (D'_{i})^{UK_{2,\tilde{x}_{aid'}}}, \tilde{D'}_{i} = (D'_{i})^{UK_{1,\tilde{x}_{aid'}}},\\
			&if \rho(i) \neq \tilde{x}_{aid'} : \tilde{C}_{i} = C_{i}, \tilde{D'}_{i} = D'_{i}\Bigl)
		\end{align*}
	\end{enumerate}
	
\end{enumerate}



\section{Secure Privacy Preserving EHR Framework}


\subsection{System Architecture/Model}

\subsubsection{System Entities}

\begin{enumerate}[label=(\arabic*)]
	\item Data Owner (DO) - The DO is responsible for the generation of data to be stored on the platform and retains ownership of data for the duration of its life cycle. The Data Owner also specifies what requirements are to be met for any user to have access by indicating what attributes they should possess though the policy used for data encryption.
	
	\item Data User (DU) - The DU represents entities that require access to stored data. They are required to possess the right attributes in order to gain access to stored data on the cloud.
	
	\item Certificate Authority (CA) - The CA plays the central role of assisting with the verification of the identity of the different entities to enable them to securely communicate with one another. It does this by providing the corresponding certificates and public parameters used in the verification process.
	
	\item Attribute Authority (AA) - The AA is responsible for providing the appropriate public and private key parameters for the different attributes under their control to the appropriate owners and users for the encryption and decryption of data. 
	
	\item Cloud Service Provider (CSP) - This entity provides the necessary storage facilities for data on the platform and also the processing capabilities for part of the operation related to the decryption of data.
	
\end{enumerate}

- system architecture diagram

\begin{figure}[h]
	\centering
	\includegraphics[scale=.50]{"System Architecture Diagram (Draft 1)"}
	\caption{System Architecture Diagram}
	\label{fig:sysarch}
\end{figure}

Figure \ref{fig:sysarch} shows the complete architecture of the entire system framework, indicating the various entities and the different ways in which they interact with each other as part of the overall system.


\subsubsection{Use Case}

- sequence diagram capturing roles involved in the use case
- Players
\begin{itemize}
	\item Data Owners - Hospitals
	\item Data Users - Other Hospitals, Universities and other research institutes,etc
	\item AAs - Hospital Board(REB + Management), University Boards(REB and Management), University Departments
	\item CSP - AWS, Google Cloud, etc
	\item CA - Local and Provincial Authorities related to health care and privacy
\end{itemize}


\begin{figure}[h]
	\centering
	\includegraphics[width=\textwidth]{"System Sequence Diagram"}
	\caption{System Sequence Diagram}
	\label{fig:sysseq}
\end{figure}

The sequence diagram is displayed in figure \ref{fig:sysseq} showing a full process from system setup to encryption and decryption.