\chapter{Proposed Solution}
\label{chap:proposed_solution}

\renewcommand{\theenumi}{\Alph{enumi}}
\renewcommand{\theenumii}{\roman{enumii}}

\section{Overview}

\begin{itemize}
	\item what the chapter will contain: description of the proposed framework for secure storage of data using a combination of attribute and symmetric-based encryption.  details of implementation. Also provide info on the security model for the underlying attribute based encryption scheme. This would highlight the assumptions that was used to verifying the integrity of the proposed scheme. I will summarize the pros and cons of the proposed scheme.
	
	\item what are the good things about the model: its more secure than xyz, runs faster. potential drawbacks: ... \textcolor{red}{its own section?}
	
	\item what the subsequent sections will contain to provide a clearer image of their content
\end{itemize}

\begin{itemize}
	\item General overview of solution at a high level? \textcolor{red}{not exactly sure of the specifics of this at this point}
\end{itemize}

\section{Proposed System Model}

\begin{itemize}
	\item describe the different entities that make up the system and the roles they play
	\item show system architecture diagram
	\item Process diagrams also
\end{itemize}

\begin{figure}[h]
	\centering
	\includegraphics[scale=.50]{"System Architecture Diagram (Draft 1)"}
	\caption{System Architecture Diagram}
	\label{fig:sysarch}
\end{figure}

In the figure \ref{fig:sysarch} the different entities that are part of the architecture are displayed with their interactions.

\begin{figure}[h]
	\centering
	\includegraphics[width=\textwidth]{"Schematic Diagram"}
	\caption{System Schematic Diagram}
	\label{fig:sysschem}
\end{figure}

The schematic diagram is displayed in figure \ref{fig:sysschem} showing a full process from system setup to encryption and decryption.


\section{Security Model}

\begin{itemize}
	\item Security game? N/B Based on security games and also has to do with interactions to verify security of scheme between an attacker and a challenger. More details from reference materials
	\item What metrics from a security point of view is the system based on? CPA vs CCA
	\item What assumption(s) is from the point of view of security is the system to be analyzed by (\textcolor{red}{to be adopted from original paper + possible outsourcing papers also} and the basic correctness and the completeness high level assessments)
\end{itemize}


\section{Framework Definition}

\begin{enumerate}
	
	\item CA Setup - This procedure is run by the CA to setup the system $$ CASetup(1^{\lambda}) \longrightarrow GMK, GPP $$
	
	where $ 1^{\lambda} $ is the security parameter, $ GMK $ is the Global Master Key, and $ GPP $ represents the Global Public Parameters.
	
	The CA chooses two multiplicative groups $ G, G_{T} $ with the same prime order $ p $ and a bilinear map $ e \colon G \times G \rightarrow G_{T} $ and a hash function $ H \colon \{0, 1\} \rightarrow G. $ The CA chooses two random numbers $ a, b \in Z_{p} $
	
	The GMK is set as $ (a, b) $ and the GPP is set to $ (g, g^{a}, g^{b}, H) $
	
	\begin{enumerate}
		
		\item User Registration
		
		This is executed for all users in the system by the CA. Each user is assigned a globally unique $ uid $ and for each $ uid $, the CA generates two random numbers $ u_{uid}, u'_{uid} \in Z_{p} $ as its global secret keys
		\begin{align*}
			GSK_{uid} = u_{uid}, GSK'_{uid} = u'_{uid}
		\end{align*}
		The global public keys for each user is generated as 
		\begin{align*}
			GPK_{uid} = g^{u_{uid}}, GPK'_{uid} = g^{\frac{1}{u'_{uid}}}
		\end{align*}
		The CA generates a $ Certificate{uid} $ for each user $ uid $ and send $ (GPK_{uid}, GSK'_{uid}, \\ Cerificate(uid)) $ to the user.
		
		\item Attribute Authority (AA) Registration
		
		The CA assigns a globally unique authority identity $ aid $ to each AA and sends $ (GPK'_{uid}, GSK_{uid}) $ for all registered users to $ AA_{aid} $. The CA also sends the verification key $ vk_{CA} $ to $ AA_{aid} $ for verifying the $ Certificate(uid) $ assigned to each user.
		
	\end{enumerate}
	
	\item AA Setup
	
	Let $ X_{aid} $ be the set of all attributes managed by $ AA_{aid} $. The AA chooses three random numbers $ \alpha_{aid}, \beta_{aid}, \gamma_{aid} \in Z_{p} $ as its secret key.
	\begin{align*}
		SK_{aid} = (\alpha_{aid}, \beta_{aid}, \gamma_{aid}) \\
		PK_{aid} = (e(g, g)^{\alpha_{aid}}, g^{\beta_{aid}}, g^{\frac{1}{\beta_{aid}}})
	\end{align*}
	For each attribute $ x_{aid} \in X_{aid} $, $ AA_{aid} $ generates a public attribute key as $ PK_{x_{aid}} = (PK_{1, x_{aid}} = H(x_{aid})^{v_{x_{aid}}}, PK_{2, x_{aid}} = H(x_{aid})^{v_{x_{aid}}\gamma_{aid}}) $ where $ v_{x_{aid}} $ is the version key of attribute $ x_{aid} $ i.e $ VK_{x_{aid}} = v_{x_{aid}} $.
	
	\item Secret Key Generation
	
	The Attribute Authority with $ AA_{aid} $ assigns a set of attributes $ S_{uid,aid} $ to user with $ uid $ after authentication and certificate verification. The AA chooses a random number $ t_{uid,aid} \in Z_{p} $ and computes a secret key for the user as
	\begin{align*}
		SK_{uid,aid} = (K_{uid,aid} = g^{\frac{\alpha_{aid}}{u'_{uid}}}g^{au_{uid}}g^{bt_{uid,aid}}, K'_{uid,aid} = g^{t_{uid,aid}}, \\
		\forall x_{aid} \in S_{uid,aid} \colon K_{x_{aid},uid} = g^{t_{uid,aid}\beta_{aid}}H(x_{aid})^{v_{x_{aid}}\beta_{aid}(u_{uid} + \gamma_{aid})})
	\end{align*}
	If the user $ uid $ does not hold any attributes from $ AA_{aid} $, the user secret key $ SK_{uid,aid} $ only contains $ K_{uid,aid} $.
	
	\item Data Encryption
	
	Data is divided into components based on the level of granularity required for access control i.e $ m = \{m_{i}, \cdots, m_{n}\} $. Data components are encrypted using symmetric encryption keys $ \kappa = \{\kappa_{i}, \cdots, \kappa_{n}\} $. An access structure $ (M_{k}, \rho) $ is defined for each content key $ \kappa_{i} (i = 1, \cdots, n) $ and encrypted using the ABE scheme to produce the corresponding ciphertext. Let M be an $\ell$ x n matrix, where $\ell$ denotes the total number of all the attributes and the function $ \rho $ associates rows of M to attributes. The function $\rho$ is not required to be injective which allows for an attribute to be associated with more than one row of M.
	
	To encrypt the content key $\kappa_{i}$, the algorithm chooses a random element $ s \in Z_{p} $ which is used as the random encryption exponent. It then selects a random vector $ \vec{v} = (s, y_{2},\ldots,y_{n}) \in Z_{p} $ where $ y_{2},\ldots,y_{n} $ are used to share the encryption exponent s. It then computes $ \forall 1 \leq i \leq \ell: \lambda_{i} = \vec{v}.M_{i} $ where $ M_{i} $ is the vector corresponding to the i-th row of M. It then randomly selects $ r_{1},r_{2},\ldots,r_{\ell} $ and computes the ciphertext as	
	\begin{align*}
		&CT_{K_{i}} = \biggl(C = K_{i}\cdot\Bigl(\prod\limits_{aid_{k} \in I_{A}}PK_{aid_{k}}\Bigl)^{s}, C' = g^{s}, C'' = g^{bs},\\
		&\forall 1 \leq i \leq \ell, \rho(i) \in X_{aid_{k}} \colon C_{i} = g^{a\lambda_{i}\cdot\bigl(PK_{i,\rho(i)}\bigl)^{-r_{i}}}, C'_{i} = g^{r_{i}},D_{i} = g^{\frac{r_{i}}{\beta_{aid_k}}}, D'_{i} = \Bigl(PK_{2,\rho_{(i)}}\Bigl)^{r_{i}}\biggl)
	\end{align*}
	Then the encrypted data is uploaded to the cloud server by the owner.
	
	\item Data Decryption
	$\kappa$
	\begin{enumerate}
		
		\item Phase 1 - Token Generation\\
		This is the first phase of the data encryption component that involves the transformation of the ciphertext into an El Gamal style ciphertext referred to in this framework as a token through partial decryption while the integrity of the original message is preserved.
		\begin{align*}
			TK = \prod\limits_{aid_{k} \in I_{A}} \frac{e\Bigl(C', K_{uid,aid}\Bigl)e\Bigl(C'', K'_{uid,aid}\Bigl)^{-1}}{\prod\limits_{i \in I_{aid_k}}\biggl(e\Bigl(C_i, GPK_{uid}\Bigl)e\Bigl(D_i, K_{\rho(i),uid}\Bigl)e\Bigl(C'_i, K_{uid,aid_k}^{-1}\Bigl)e\Bigl(g, D'_i\Bigl)^{-1}\biggl)^{\omega_{i}n_A}}
		\end{align*}
		N/B - 
		
		\textcolor{red}{Look up the description of the process for the generation of the secret share using the shares represented by the attributes that are contained in the key. This technique has something to do with matrices. For potentially more details look up the papers by brent waters or allison lewko? for additional information in a more summarized format. For more intense details look at the PhD thesis of ben lynn which is the main source}
		
		$\omega_{i}$ - constant chosen for the reconstruction of encryption exponent s using shares $\lambda_{i}$
		
		$n_A$ - number of AAs involved in the ciphertext
		
		$\rho(i)$ - mapping of each row of access structure to attribute i
		\begin{align*}
			TK = \prod\limits_{aid_{k} \in I_{A}} e\Bigl(g, g\Bigl)^{\frac{s\alpha_{aid}}{u'_{uid}}}
		\end{align*}
		\item Phase 2 - Decryption
		
		The user does an exponentiation on the token to get the blinding element (BE) \textcolor{red}{[rephrase this later]} for decryption.
		\begin{align*}
			BE= TK^{u'_{uid}} = \prod\limits_{aid_{k} \in I_{A}} e\Bigl(g, g\Bigl)^{s\alpha_{aid}}
		\end{align*}
		Remember the $C$ element of the ciphertext $= K_i\cdot\bigl(\prod\limits_{aid_{k} \in I_{A}}PK_{aid_{k}}\bigl)^{s}$ where $PK_{aid_{k}} = e\bigl(g, g\bigl)^{\alpha_{aid}}$. Therefore the original message which in this case is the symmetric key is computed as
		\begin{align*}
			K = \frac{C}{BE}
		\end{align*}
	\end{enumerate}

	\item User Revocation
	
	\begin{enumerate}
		\item Key Update
		
		\item Ciphertext Update
	\end{enumerate}
	
\end{enumerate}
