\addcontentsline{toc}{chapter}{Abstract}
\chapter*{Abstract}

The need to protect user data from unauthorized access and malicious use by authorized users, especially in the case of a system that includes the use of a third party storage service, which limits the amount of control that the data owner has, continues to present itself. The use of encryption for secure data storage has continued to evolve in order to meet the need for flexible and fine grained access control which led to the development of Attribute Based Encryption (ABE) based on the concept of Identity Based Encryption (IBE). ABE gives more control to the data owner and has continued to evolve to enable users make use of the technological advancements available to them.

The use of ABE to ensure the security and privacy of health data has been explored with multiple solutions proposed. This thesis aims to develop a platform that applies an improved ABE scheme which allows for the secure outsourcing the more computationally intensive processes for data decryption to the cloud servers, reducing the amount of time needed for decryption to occur at the user end and also reducing the amount of computational power needed by users who require access to the data stored in the cloud in its encrypted form.

% This proposal is aimed towards providing more insight on the different ABE schemes that have been developed, highlighting the additional features that have been added such as the use of multiple authorities, the ability to revoke secret keys and accountability in order to identify key abuse scenarios. Furthermore, it highlights the various systems that have been created using ABE in an attempt to meet the ever increasing demand for privacy and security in today's Information Technology infrastructure. Information is provided about the proposed system that is to be developed and the possible additional features that could be implemented to give the system some level of novelty in comparison to already existing works.